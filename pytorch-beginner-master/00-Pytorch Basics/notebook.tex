
% Default to the notebook output style

    


% Inherit from the specified cell style.




    
\documentclass[11pt]{article}

    
    
    \usepackage[T1]{fontenc}
    % Nicer default font (+ math font) than Computer Modern for most use cases
    \usepackage{mathpazo}

    % Basic figure setup, for now with no caption control since it's done
    % automatically by Pandoc (which extracts ![](path) syntax from Markdown).
    \usepackage{graphicx}
    % We will generate all images so they have a width \maxwidth. This means
    % that they will get their normal width if they fit onto the page, but
    % are scaled down if they would overflow the margins.
    \makeatletter
    \def\maxwidth{\ifdim\Gin@nat@width>\linewidth\linewidth
    \else\Gin@nat@width\fi}
    \makeatother
    \let\Oldincludegraphics\includegraphics
    % Set max figure width to be 80% of text width, for now hardcoded.
    \renewcommand{\includegraphics}[1]{\Oldincludegraphics[width=.8\maxwidth]{#1}}
    % Ensure that by default, figures have no caption (until we provide a
    % proper Figure object with a Caption API and a way to capture that
    % in the conversion process - todo).
    \usepackage{caption}
    \DeclareCaptionLabelFormat{nolabel}{}
    \captionsetup{labelformat=nolabel}

    \usepackage{adjustbox} % Used to constrain images to a maximum size 
    \usepackage{xcolor} % Allow colors to be defined
    \usepackage{enumerate} % Needed for markdown enumerations to work
    \usepackage{geometry} % Used to adjust the document margins
    \usepackage{amsmath} % Equations
    \usepackage{amssymb} % Equations
    \usepackage{textcomp} % defines textquotesingle
    % Hack from http://tex.stackexchange.com/a/47451/13684:
    \AtBeginDocument{%
        \def\PYZsq{\textquotesingle}% Upright quotes in Pygmentized code
    }
    \usepackage{upquote} % Upright quotes for verbatim code
    \usepackage{eurosym} % defines \euro
    \usepackage[mathletters]{ucs} % Extended unicode (utf-8) support
    \usepackage[utf8x]{inputenc} % Allow utf-8 characters in the tex document
    \usepackage{fancyvrb} % verbatim replacement that allows latex
    \usepackage{grffile} % extends the file name processing of package graphics 
                         % to support a larger range 
    % The hyperref package gives us a pdf with properly built
    % internal navigation ('pdf bookmarks' for the table of contents,
    % internal cross-reference links, web links for URLs, etc.)
    \usepackage{hyperref}
    \usepackage{longtable} % longtable support required by pandoc >1.10
    \usepackage{booktabs}  % table support for pandoc > 1.12.2
    \usepackage[inline]{enumitem} % IRkernel/repr support (it uses the enumerate* environment)
    \usepackage[normalem]{ulem} % ulem is needed to support strikethroughs (\sout)
                                % normalem makes italics be italics, not underlines
    

    
    
    % Colors for the hyperref package
    \definecolor{urlcolor}{rgb}{0,.145,.698}
    \definecolor{linkcolor}{rgb}{.71,0.21,0.01}
    \definecolor{citecolor}{rgb}{.12,.54,.11}

    % ANSI colors
    \definecolor{ansi-black}{HTML}{3E424D}
    \definecolor{ansi-black-intense}{HTML}{282C36}
    \definecolor{ansi-red}{HTML}{E75C58}
    \definecolor{ansi-red-intense}{HTML}{B22B31}
    \definecolor{ansi-green}{HTML}{00A250}
    \definecolor{ansi-green-intense}{HTML}{007427}
    \definecolor{ansi-yellow}{HTML}{DDB62B}
    \definecolor{ansi-yellow-intense}{HTML}{B27D12}
    \definecolor{ansi-blue}{HTML}{208FFB}
    \definecolor{ansi-blue-intense}{HTML}{0065CA}
    \definecolor{ansi-magenta}{HTML}{D160C4}
    \definecolor{ansi-magenta-intense}{HTML}{A03196}
    \definecolor{ansi-cyan}{HTML}{60C6C8}
    \definecolor{ansi-cyan-intense}{HTML}{258F8F}
    \definecolor{ansi-white}{HTML}{C5C1B4}
    \definecolor{ansi-white-intense}{HTML}{A1A6B2}

    % commands and environments needed by pandoc snippets
    % extracted from the output of `pandoc -s`
    \providecommand{\tightlist}{%
      \setlength{\itemsep}{0pt}\setlength{\parskip}{0pt}}
    \DefineVerbatimEnvironment{Highlighting}{Verbatim}{commandchars=\\\{\}}
    % Add ',fontsize=\small' for more characters per line
    \newenvironment{Shaded}{}{}
    \newcommand{\KeywordTok}[1]{\textcolor[rgb]{0.00,0.44,0.13}{\textbf{{#1}}}}
    \newcommand{\DataTypeTok}[1]{\textcolor[rgb]{0.56,0.13,0.00}{{#1}}}
    \newcommand{\DecValTok}[1]{\textcolor[rgb]{0.25,0.63,0.44}{{#1}}}
    \newcommand{\BaseNTok}[1]{\textcolor[rgb]{0.25,0.63,0.44}{{#1}}}
    \newcommand{\FloatTok}[1]{\textcolor[rgb]{0.25,0.63,0.44}{{#1}}}
    \newcommand{\CharTok}[1]{\textcolor[rgb]{0.25,0.44,0.63}{{#1}}}
    \newcommand{\StringTok}[1]{\textcolor[rgb]{0.25,0.44,0.63}{{#1}}}
    \newcommand{\CommentTok}[1]{\textcolor[rgb]{0.38,0.63,0.69}{\textit{{#1}}}}
    \newcommand{\OtherTok}[1]{\textcolor[rgb]{0.00,0.44,0.13}{{#1}}}
    \newcommand{\AlertTok}[1]{\textcolor[rgb]{1.00,0.00,0.00}{\textbf{{#1}}}}
    \newcommand{\FunctionTok}[1]{\textcolor[rgb]{0.02,0.16,0.49}{{#1}}}
    \newcommand{\RegionMarkerTok}[1]{{#1}}
    \newcommand{\ErrorTok}[1]{\textcolor[rgb]{1.00,0.00,0.00}{\textbf{{#1}}}}
    \newcommand{\NormalTok}[1]{{#1}}
    
    % Additional commands for more recent versions of Pandoc
    \newcommand{\ConstantTok}[1]{\textcolor[rgb]{0.53,0.00,0.00}{{#1}}}
    \newcommand{\SpecialCharTok}[1]{\textcolor[rgb]{0.25,0.44,0.63}{{#1}}}
    \newcommand{\VerbatimStringTok}[1]{\textcolor[rgb]{0.25,0.44,0.63}{{#1}}}
    \newcommand{\SpecialStringTok}[1]{\textcolor[rgb]{0.73,0.40,0.53}{{#1}}}
    \newcommand{\ImportTok}[1]{{#1}}
    \newcommand{\DocumentationTok}[1]{\textcolor[rgb]{0.73,0.13,0.13}{\textit{{#1}}}}
    \newcommand{\AnnotationTok}[1]{\textcolor[rgb]{0.38,0.63,0.69}{\textbf{\textit{{#1}}}}}
    \newcommand{\CommentVarTok}[1]{\textcolor[rgb]{0.38,0.63,0.69}{\textbf{\textit{{#1}}}}}
    \newcommand{\VariableTok}[1]{\textcolor[rgb]{0.10,0.09,0.49}{{#1}}}
    \newcommand{\ControlFlowTok}[1]{\textcolor[rgb]{0.00,0.44,0.13}{\textbf{{#1}}}}
    \newcommand{\OperatorTok}[1]{\textcolor[rgb]{0.40,0.40,0.40}{{#1}}}
    \newcommand{\BuiltInTok}[1]{{#1}}
    \newcommand{\ExtensionTok}[1]{{#1}}
    \newcommand{\PreprocessorTok}[1]{\textcolor[rgb]{0.74,0.48,0.00}{{#1}}}
    \newcommand{\AttributeTok}[1]{\textcolor[rgb]{0.49,0.56,0.16}{{#1}}}
    \newcommand{\InformationTok}[1]{\textcolor[rgb]{0.38,0.63,0.69}{\textbf{\textit{{#1}}}}}
    \newcommand{\WarningTok}[1]{\textcolor[rgb]{0.38,0.63,0.69}{\textbf{\textit{{#1}}}}}
    
    
    % Define a nice break command that doesn't care if a line doesn't already
    % exist.
    \def\br{\hspace*{\fill} \\* }
    % Math Jax compatability definitions
    \def\gt{>}
    \def\lt{<}
    % Document parameters
    \title{basic}
    
    
    

    % Pygments definitions
    
\makeatletter
\def\PY@reset{\let\PY@it=\relax \let\PY@bf=\relax%
    \let\PY@ul=\relax \let\PY@tc=\relax%
    \let\PY@bc=\relax \let\PY@ff=\relax}
\def\PY@tok#1{\csname PY@tok@#1\endcsname}
\def\PY@toks#1+{\ifx\relax#1\empty\else%
    \PY@tok{#1}\expandafter\PY@toks\fi}
\def\PY@do#1{\PY@bc{\PY@tc{\PY@ul{%
    \PY@it{\PY@bf{\PY@ff{#1}}}}}}}
\def\PY#1#2{\PY@reset\PY@toks#1+\relax+\PY@do{#2}}

\expandafter\def\csname PY@tok@gd\endcsname{\def\PY@tc##1{\textcolor[rgb]{0.63,0.00,0.00}{##1}}}
\expandafter\def\csname PY@tok@gu\endcsname{\let\PY@bf=\textbf\def\PY@tc##1{\textcolor[rgb]{0.50,0.00,0.50}{##1}}}
\expandafter\def\csname PY@tok@gt\endcsname{\def\PY@tc##1{\textcolor[rgb]{0.00,0.27,0.87}{##1}}}
\expandafter\def\csname PY@tok@gs\endcsname{\let\PY@bf=\textbf}
\expandafter\def\csname PY@tok@gr\endcsname{\def\PY@tc##1{\textcolor[rgb]{1.00,0.00,0.00}{##1}}}
\expandafter\def\csname PY@tok@cm\endcsname{\let\PY@it=\textit\def\PY@tc##1{\textcolor[rgb]{0.25,0.50,0.50}{##1}}}
\expandafter\def\csname PY@tok@vg\endcsname{\def\PY@tc##1{\textcolor[rgb]{0.10,0.09,0.49}{##1}}}
\expandafter\def\csname PY@tok@vi\endcsname{\def\PY@tc##1{\textcolor[rgb]{0.10,0.09,0.49}{##1}}}
\expandafter\def\csname PY@tok@vm\endcsname{\def\PY@tc##1{\textcolor[rgb]{0.10,0.09,0.49}{##1}}}
\expandafter\def\csname PY@tok@mh\endcsname{\def\PY@tc##1{\textcolor[rgb]{0.40,0.40,0.40}{##1}}}
\expandafter\def\csname PY@tok@cs\endcsname{\let\PY@it=\textit\def\PY@tc##1{\textcolor[rgb]{0.25,0.50,0.50}{##1}}}
\expandafter\def\csname PY@tok@ge\endcsname{\let\PY@it=\textit}
\expandafter\def\csname PY@tok@vc\endcsname{\def\PY@tc##1{\textcolor[rgb]{0.10,0.09,0.49}{##1}}}
\expandafter\def\csname PY@tok@il\endcsname{\def\PY@tc##1{\textcolor[rgb]{0.40,0.40,0.40}{##1}}}
\expandafter\def\csname PY@tok@go\endcsname{\def\PY@tc##1{\textcolor[rgb]{0.53,0.53,0.53}{##1}}}
\expandafter\def\csname PY@tok@cp\endcsname{\def\PY@tc##1{\textcolor[rgb]{0.74,0.48,0.00}{##1}}}
\expandafter\def\csname PY@tok@gi\endcsname{\def\PY@tc##1{\textcolor[rgb]{0.00,0.63,0.00}{##1}}}
\expandafter\def\csname PY@tok@gh\endcsname{\let\PY@bf=\textbf\def\PY@tc##1{\textcolor[rgb]{0.00,0.00,0.50}{##1}}}
\expandafter\def\csname PY@tok@ni\endcsname{\let\PY@bf=\textbf\def\PY@tc##1{\textcolor[rgb]{0.60,0.60,0.60}{##1}}}
\expandafter\def\csname PY@tok@nl\endcsname{\def\PY@tc##1{\textcolor[rgb]{0.63,0.63,0.00}{##1}}}
\expandafter\def\csname PY@tok@nn\endcsname{\let\PY@bf=\textbf\def\PY@tc##1{\textcolor[rgb]{0.00,0.00,1.00}{##1}}}
\expandafter\def\csname PY@tok@no\endcsname{\def\PY@tc##1{\textcolor[rgb]{0.53,0.00,0.00}{##1}}}
\expandafter\def\csname PY@tok@na\endcsname{\def\PY@tc##1{\textcolor[rgb]{0.49,0.56,0.16}{##1}}}
\expandafter\def\csname PY@tok@nb\endcsname{\def\PY@tc##1{\textcolor[rgb]{0.00,0.50,0.00}{##1}}}
\expandafter\def\csname PY@tok@nc\endcsname{\let\PY@bf=\textbf\def\PY@tc##1{\textcolor[rgb]{0.00,0.00,1.00}{##1}}}
\expandafter\def\csname PY@tok@nd\endcsname{\def\PY@tc##1{\textcolor[rgb]{0.67,0.13,1.00}{##1}}}
\expandafter\def\csname PY@tok@ne\endcsname{\let\PY@bf=\textbf\def\PY@tc##1{\textcolor[rgb]{0.82,0.25,0.23}{##1}}}
\expandafter\def\csname PY@tok@nf\endcsname{\def\PY@tc##1{\textcolor[rgb]{0.00,0.00,1.00}{##1}}}
\expandafter\def\csname PY@tok@si\endcsname{\let\PY@bf=\textbf\def\PY@tc##1{\textcolor[rgb]{0.73,0.40,0.53}{##1}}}
\expandafter\def\csname PY@tok@s2\endcsname{\def\PY@tc##1{\textcolor[rgb]{0.73,0.13,0.13}{##1}}}
\expandafter\def\csname PY@tok@nt\endcsname{\let\PY@bf=\textbf\def\PY@tc##1{\textcolor[rgb]{0.00,0.50,0.00}{##1}}}
\expandafter\def\csname PY@tok@nv\endcsname{\def\PY@tc##1{\textcolor[rgb]{0.10,0.09,0.49}{##1}}}
\expandafter\def\csname PY@tok@s1\endcsname{\def\PY@tc##1{\textcolor[rgb]{0.73,0.13,0.13}{##1}}}
\expandafter\def\csname PY@tok@dl\endcsname{\def\PY@tc##1{\textcolor[rgb]{0.73,0.13,0.13}{##1}}}
\expandafter\def\csname PY@tok@ch\endcsname{\let\PY@it=\textit\def\PY@tc##1{\textcolor[rgb]{0.25,0.50,0.50}{##1}}}
\expandafter\def\csname PY@tok@m\endcsname{\def\PY@tc##1{\textcolor[rgb]{0.40,0.40,0.40}{##1}}}
\expandafter\def\csname PY@tok@gp\endcsname{\let\PY@bf=\textbf\def\PY@tc##1{\textcolor[rgb]{0.00,0.00,0.50}{##1}}}
\expandafter\def\csname PY@tok@sh\endcsname{\def\PY@tc##1{\textcolor[rgb]{0.73,0.13,0.13}{##1}}}
\expandafter\def\csname PY@tok@ow\endcsname{\let\PY@bf=\textbf\def\PY@tc##1{\textcolor[rgb]{0.67,0.13,1.00}{##1}}}
\expandafter\def\csname PY@tok@sx\endcsname{\def\PY@tc##1{\textcolor[rgb]{0.00,0.50,0.00}{##1}}}
\expandafter\def\csname PY@tok@bp\endcsname{\def\PY@tc##1{\textcolor[rgb]{0.00,0.50,0.00}{##1}}}
\expandafter\def\csname PY@tok@c1\endcsname{\let\PY@it=\textit\def\PY@tc##1{\textcolor[rgb]{0.25,0.50,0.50}{##1}}}
\expandafter\def\csname PY@tok@fm\endcsname{\def\PY@tc##1{\textcolor[rgb]{0.00,0.00,1.00}{##1}}}
\expandafter\def\csname PY@tok@o\endcsname{\def\PY@tc##1{\textcolor[rgb]{0.40,0.40,0.40}{##1}}}
\expandafter\def\csname PY@tok@kc\endcsname{\let\PY@bf=\textbf\def\PY@tc##1{\textcolor[rgb]{0.00,0.50,0.00}{##1}}}
\expandafter\def\csname PY@tok@c\endcsname{\let\PY@it=\textit\def\PY@tc##1{\textcolor[rgb]{0.25,0.50,0.50}{##1}}}
\expandafter\def\csname PY@tok@mf\endcsname{\def\PY@tc##1{\textcolor[rgb]{0.40,0.40,0.40}{##1}}}
\expandafter\def\csname PY@tok@err\endcsname{\def\PY@bc##1{\setlength{\fboxsep}{0pt}\fcolorbox[rgb]{1.00,0.00,0.00}{1,1,1}{\strut ##1}}}
\expandafter\def\csname PY@tok@mb\endcsname{\def\PY@tc##1{\textcolor[rgb]{0.40,0.40,0.40}{##1}}}
\expandafter\def\csname PY@tok@ss\endcsname{\def\PY@tc##1{\textcolor[rgb]{0.10,0.09,0.49}{##1}}}
\expandafter\def\csname PY@tok@sr\endcsname{\def\PY@tc##1{\textcolor[rgb]{0.73,0.40,0.53}{##1}}}
\expandafter\def\csname PY@tok@mo\endcsname{\def\PY@tc##1{\textcolor[rgb]{0.40,0.40,0.40}{##1}}}
\expandafter\def\csname PY@tok@kd\endcsname{\let\PY@bf=\textbf\def\PY@tc##1{\textcolor[rgb]{0.00,0.50,0.00}{##1}}}
\expandafter\def\csname PY@tok@mi\endcsname{\def\PY@tc##1{\textcolor[rgb]{0.40,0.40,0.40}{##1}}}
\expandafter\def\csname PY@tok@kn\endcsname{\let\PY@bf=\textbf\def\PY@tc##1{\textcolor[rgb]{0.00,0.50,0.00}{##1}}}
\expandafter\def\csname PY@tok@cpf\endcsname{\let\PY@it=\textit\def\PY@tc##1{\textcolor[rgb]{0.25,0.50,0.50}{##1}}}
\expandafter\def\csname PY@tok@kr\endcsname{\let\PY@bf=\textbf\def\PY@tc##1{\textcolor[rgb]{0.00,0.50,0.00}{##1}}}
\expandafter\def\csname PY@tok@s\endcsname{\def\PY@tc##1{\textcolor[rgb]{0.73,0.13,0.13}{##1}}}
\expandafter\def\csname PY@tok@kp\endcsname{\def\PY@tc##1{\textcolor[rgb]{0.00,0.50,0.00}{##1}}}
\expandafter\def\csname PY@tok@w\endcsname{\def\PY@tc##1{\textcolor[rgb]{0.73,0.73,0.73}{##1}}}
\expandafter\def\csname PY@tok@kt\endcsname{\def\PY@tc##1{\textcolor[rgb]{0.69,0.00,0.25}{##1}}}
\expandafter\def\csname PY@tok@sc\endcsname{\def\PY@tc##1{\textcolor[rgb]{0.73,0.13,0.13}{##1}}}
\expandafter\def\csname PY@tok@sb\endcsname{\def\PY@tc##1{\textcolor[rgb]{0.73,0.13,0.13}{##1}}}
\expandafter\def\csname PY@tok@sa\endcsname{\def\PY@tc##1{\textcolor[rgb]{0.73,0.13,0.13}{##1}}}
\expandafter\def\csname PY@tok@k\endcsname{\let\PY@bf=\textbf\def\PY@tc##1{\textcolor[rgb]{0.00,0.50,0.00}{##1}}}
\expandafter\def\csname PY@tok@se\endcsname{\let\PY@bf=\textbf\def\PY@tc##1{\textcolor[rgb]{0.73,0.40,0.13}{##1}}}
\expandafter\def\csname PY@tok@sd\endcsname{\let\PY@it=\textit\def\PY@tc##1{\textcolor[rgb]{0.73,0.13,0.13}{##1}}}

\def\PYZbs{\char`\\}
\def\PYZus{\char`\_}
\def\PYZob{\char`\{}
\def\PYZcb{\char`\}}
\def\PYZca{\char`\^}
\def\PYZam{\char`\&}
\def\PYZlt{\char`\<}
\def\PYZgt{\char`\>}
\def\PYZsh{\char`\#}
\def\PYZpc{\char`\%}
\def\PYZdl{\char`\$}
\def\PYZhy{\char`\-}
\def\PYZsq{\char`\'}
\def\PYZdq{\char`\"}
\def\PYZti{\char`\~}
% for compatibility with earlier versions
\def\PYZat{@}
\def\PYZlb{[}
\def\PYZrb{]}
\makeatother


    % Exact colors from NB
    \definecolor{incolor}{rgb}{0.0, 0.0, 0.5}
    \definecolor{outcolor}{rgb}{0.545, 0.0, 0.0}



    
    % Prevent overflowing lines due to hard-to-break entities
    \sloppy 
    % Setup hyperref package
    \hypersetup{
      breaklinks=true,  % so long urls are correctly broken across lines
      colorlinks=true,
      urlcolor=urlcolor,
      linkcolor=linkcolor,
      citecolor=citecolor,
      }
    % Slightly bigger margins than the latex defaults
    
    \geometry{verbose,tmargin=1in,bmargin=1in,lmargin=1in,rmargin=1in}
    
    

    \begin{document}
    
    
    \maketitle
    
    

    
    \begin{Verbatim}[commandchars=\\\{\}]
{\color{incolor}In [{\color{incolor}1}]:} \PY{k+kn}{import} \PY{n+nn}{torch}
\end{Verbatim}


    \begin{Verbatim}[commandchars=\\\{\}]
{\color{incolor}In [{\color{incolor}5}]:} \PY{k+kn}{import} \PY{n+nn}{numpy} \PY{k}{as} \PY{n+nn}{np}
\end{Verbatim}


    \subsubsection{torch
基本处理单元}\label{torch-ux57faux672cux5904ux7406ux5355ux5143}

    \begin{Verbatim}[commandchars=\\\{\}]
{\color{incolor}In [{\color{incolor}4}]:} \PY{c+c1}{\PYZsh{} 返回的数组大小5x4的矩阵}
        \PY{n}{torch}\PY{o}{.}\PY{n}{Tensor}\PY{p}{(}\PY{l+m+mi}{5}\PY{p}{,} \PY{l+m+mi}{4}\PY{p}{)}
\end{Verbatim}


\begin{Verbatim}[commandchars=\\\{\}]
{\color{outcolor}Out[{\color{outcolor}4}]:} 
        1.00000e-03 *
         -6.7953  0.0000 -6.7953  0.0000
          0.0000  0.0000  0.0000  0.0000
          0.0000  0.0000  0.0000  0.0000
          0.0000  0.0000  0.0000  0.0000
          0.0000  0.0000  0.0000  0.0000
        [torch.FloatTensor of size 5x4]
\end{Verbatim}
            
    \begin{Verbatim}[commandchars=\\\{\}]
{\color{incolor}In [{\color{incolor}10}]:} \PY{c+c1}{\PYZsh{} 返回的数组大小是5x4的矩阵,初始化是0\PYZti{}1的均匀分布}
         \PY{n}{torch}\PY{o}{.}\PY{n}{rand}\PY{p}{(}\PY{l+m+mi}{5}\PY{p}{,} \PY{l+m+mi}{4}\PY{p}{)}
\end{Verbatim}


\begin{Verbatim}[commandchars=\\\{\}]
{\color{outcolor}Out[{\color{outcolor}10}]:} 
          0.7977  0.1654  0.8409  0.7564
          0.3139  0.2253  0.6559  0.2096
          0.3448  0.7688  0.8343  0.9917
          0.2122  0.1920  0.1087  0.3080
          0.9847  0.0374  0.2029  0.7608
         [torch.FloatTensor of size 5x4]
\end{Verbatim}
            
    \begin{Verbatim}[commandchars=\\\{\}]
{\color{incolor}In [{\color{incolor}13}]:} \PY{c+c1}{\PYZsh{} 得到矩阵大小}
         \PY{n}{a} \PY{o}{=} \PY{n}{torch}\PY{o}{.}\PY{n}{rand}\PY{p}{(}\PY{l+m+mi}{5}\PY{p}{,} \PY{l+m+mi}{4}\PY{p}{)}
         \PY{n}{a}\PY{o}{.}\PY{n}{size}\PY{p}{(}\PY{p}{)}
\end{Verbatim}


\begin{Verbatim}[commandchars=\\\{\}]
{\color{outcolor}Out[{\color{outcolor}13}]:} torch.Size([5, 4])
\end{Verbatim}
            
    \begin{Verbatim}[commandchars=\\\{\}]
{\color{incolor}In [{\color{incolor}12}]:} \PY{c+c1}{\PYZsh{} numpy 类似的返回5x4大小的矩阵}
         \PY{n}{np}\PY{o}{.}\PY{n}{ones}\PY{p}{(}\PY{p}{(}\PY{l+m+mi}{5}\PY{p}{,} \PY{l+m+mi}{4}\PY{p}{)}\PY{p}{)}
\end{Verbatim}


\begin{Verbatim}[commandchars=\\\{\}]
{\color{outcolor}Out[{\color{outcolor}12}]:} array([[ 1.,  1.,  1.,  1.],
                [ 1.,  1.,  1.,  1.],
                [ 1.,  1.,  1.,  1.],
                [ 1.,  1.,  1.,  1.],
                [ 1.,  1.,  1.,  1.]])
\end{Verbatim}
            
    \begin{Verbatim}[commandchars=\\\{\}]
{\color{incolor}In [{\color{incolor}24}]:} \PY{c+c1}{\PYZsh{} numpy 和 torch.Tensor 之间的转换}
         \PY{n}{a} \PY{o}{=} \PY{n}{torch}\PY{o}{.}\PY{n}{rand}\PY{p}{(}\PY{l+m+mi}{5}\PY{p}{,} \PY{l+m+mi}{4}\PY{p}{)}
         \PY{n}{b} \PY{o}{=} \PY{n}{a}\PY{o}{.}\PY{n}{numpy}\PY{p}{(}\PY{p}{)}
         \PY{n+nb}{print}\PY{p}{(}\PY{n}{b}\PY{p}{)}
\end{Verbatim}


    \begin{Verbatim}[commandchars=\\\{\}]
[[  7.89588690e-01   7.43446290e-01   1.61940977e-01   4.80842412e-01]
 [  1.78743862e-02   1.91753581e-01   3.34077984e-01   3.44260454e-01]
 [  3.15251201e-01   1.37171671e-01   3.29199225e-01   1.57159448e-01]
 [  7.46150970e-01   8.45601916e-01   9.45035994e-01   4.01963502e-01]
 [  4.72619027e-01   1.90935910e-01   8.99520470e-04   3.92095864e-01]]

    \end{Verbatim}

    \begin{Verbatim}[commandchars=\\\{\}]
{\color{incolor}In [{\color{incolor}23}]:} \PY{n}{a} \PY{o}{=} \PY{n}{np}\PY{o}{.}\PY{n}{array}\PY{p}{(}\PY{p}{[}\PY{p}{[}\PY{l+m+mi}{3}\PY{p}{,} \PY{l+m+mi}{4}\PY{p}{]}\PY{p}{,} \PY{p}{[}\PY{l+m+mi}{3}\PY{p}{,} \PY{l+m+mi}{6}\PY{p}{]}\PY{p}{]}\PY{p}{)}
         \PY{n}{b} \PY{o}{=} \PY{n}{torch}\PY{o}{.}\PY{n}{from\PYZus{}numpy}\PY{p}{(}\PY{n}{a}\PY{p}{)}
         \PY{n+nb}{print}\PY{p}{(}\PY{n}{b}\PY{p}{)}
\end{Verbatim}


    \begin{Verbatim}[commandchars=\\\{\}]

 3  4
 3  6
[torch.LongTensor of size 2x2]


    \end{Verbatim}

    运算和numpy类似

    \begin{Verbatim}[commandchars=\\\{\}]
{\color{incolor}In [{\color{incolor}14}]:} \PY{n}{x} \PY{o}{=} \PY{n}{torch}\PY{o}{.}\PY{n}{rand}\PY{p}{(}\PY{l+m+mi}{5}\PY{p}{,} \PY{l+m+mi}{4}\PY{p}{)}
         \PY{n}{y} \PY{o}{=} \PY{n}{torch}\PY{o}{.}\PY{n}{rand}\PY{p}{(}\PY{l+m+mi}{5}\PY{p}{,} \PY{l+m+mi}{4}\PY{p}{)}
         \PY{n}{c} \PY{o}{=} \PY{l+m+mi}{3}
\end{Verbatim}


    \begin{Verbatim}[commandchars=\\\{\}]
{\color{incolor}In [{\color{incolor}15}]:} \PY{n+nb}{print}\PY{p}{(}\PY{n}{c} \PY{o}{*} \PY{n}{x}\PY{p}{)}
\end{Verbatim}


    \begin{Verbatim}[commandchars=\\\{\}]

 1.8998  2.4494  0.4745  2.7200
 2.7077  2.6774  2.5322  1.7018
 2.1523  1.3062  2.9104  0.8672
 2.3793  1.4136  2.4238  0.4781
 0.3026  0.1418  0.9806  2.1584
[torch.FloatTensor of size 5x4]


    \end{Verbatim}

    \begin{Verbatim}[commandchars=\\\{\}]
{\color{incolor}In [{\color{incolor}17}]:} \PY{n+nb}{print}\PY{p}{(}\PY{n}{x} \PY{o}{+} \PY{n}{y}\PY{p}{)}
\end{Verbatim}


    \begin{Verbatim}[commandchars=\\\{\}]

 1.4478  1.1207  0.7065  1.4593
 1.5152  1.4361  1.8031  0.5893
 0.9860  0.4821  1.2809  0.5736
 1.6838  1.1084  0.9404  0.7543
 0.1237  0.1471  0.8376  1.2901
[torch.FloatTensor of size 5x4]


    \end{Verbatim}

    \begin{Verbatim}[commandchars=\\\{\}]
{\color{incolor}In [{\color{incolor}18}]:} \PY{n+nb}{print}\PY{p}{(}\PY{n}{x}\PY{o}{.}\PY{n}{add}\PY{p}{(}\PY{n}{y}\PY{p}{)}\PY{p}{)}
\end{Verbatim}


    \begin{Verbatim}[commandchars=\\\{\}]

 1.4478  1.1207  0.7065  1.4593
 1.5152  1.4361  1.8031  0.5893
 0.9860  0.4821  1.2809  0.5736
 1.6838  1.1084  0.9404  0.7543
 0.1237  0.1471  0.8376  1.2901
[torch.FloatTensor of size 5x4]


    \end{Verbatim}

    \begin{Verbatim}[commandchars=\\\{\}]
{\color{incolor}In [{\color{incolor}19}]:} \PY{c+c1}{\PYZsh{} 可以直接进行操作改变原对象,x+y或者x.add()并不会改变x,但是x.add\PYZus{}()则会对x进行改变}
         \PY{n}{x}\PY{o}{.}\PY{n}{add\PYZus{}}\PY{p}{(}\PY{n}{y}\PY{p}{)}
\end{Verbatim}


\begin{Verbatim}[commandchars=\\\{\}]
{\color{outcolor}Out[{\color{outcolor}19}]:} 
          1.4478  1.1207  0.7065  1.4593
          1.5152  1.4361  1.8031  0.5893
          0.9860  0.4821  1.2809  0.5736
          1.6838  1.1084  0.9404  0.7543
          0.1237  0.1471  0.8376  1.2901
         [torch.FloatTensor of size 5x4]
\end{Verbatim}
            
    \begin{Verbatim}[commandchars=\\\{\}]
{\color{incolor}In [{\color{incolor}20}]:} \PY{n+nb}{print}\PY{p}{(}\PY{n}{x}\PY{p}{)}
\end{Verbatim}


    \begin{Verbatim}[commandchars=\\\{\}]

 1.4478  1.1207  0.7065  1.4593
 1.5152  1.4361  1.8031  0.5893
 0.9860  0.4821  1.2809  0.5736
 1.6838  1.1084  0.9404  0.7543
 0.1237  0.1471  0.8376  1.2901
[torch.FloatTensor of size 5x4]


    \end{Verbatim}

    将 torch.Tensor 放到 GPU 上

    \begin{Verbatim}[commandchars=\\\{\}]
{\color{incolor}In [{\color{incolor}26}]:} \PY{c+c1}{\PYZsh{} 判断一下电脑是否支持GPU}
         \PY{n}{torch}\PY{o}{.}\PY{n}{cuda}\PY{o}{.}\PY{n}{is\PYZus{}available}\PY{p}{(}\PY{p}{)}
\end{Verbatim}


\begin{Verbatim}[commandchars=\\\{\}]
{\color{outcolor}Out[{\color{outcolor}26}]:} True
\end{Verbatim}
            
    \begin{Verbatim}[commandchars=\\\{\}]
{\color{incolor}In [{\color{incolor}28}]:} \PY{n}{a} \PY{o}{=} \PY{n}{torch}\PY{o}{.}\PY{n}{rand}\PY{p}{(}\PY{l+m+mi}{5}\PY{p}{,} \PY{l+m+mi}{4}\PY{p}{)}
         \PY{n}{a} \PY{o}{=} \PY{n}{a}\PY{o}{.}\PY{n}{cuda}\PY{p}{(}\PY{p}{)}
         \PY{n+nb}{print}\PY{p}{(}\PY{n}{a}\PY{p}{)}
\end{Verbatim}


    \begin{Verbatim}[commandchars=\\\{\}]

 0.6592  0.4923  0.6790  0.8395
 0.5175  0.1796  0.4245  0.1928
 0.1701  0.1957  0.3858  0.8376
 0.4047  0.1358  0.2725  0.5413
 0.7260  0.6121  0.6860  0.9933
[torch.cuda.FloatTensor of size 5x4 (GPU 0)]


    \end{Verbatim}

    \subsubsection{torch
的自动求导功能}\label{torch-ux7684ux81eaux52a8ux6c42ux5bfcux529fux80fd}

torch 和大部分框架一样有着自动求导功能,对象不再是
torch.Tensor,而是torch.autograd.Variable

本质上Variable和Tensor没有什么区别,不过Variable会放在一个计算图里面,可以进行前向传播和反向传播以及求导

\begin{figure}
\centering
\includegraphics{http://upload-images.jianshu.io/upload_images/3623720-1c2694b72e0341ce.png?imageMogr2/auto-orient/strip\%7CimageView2/2/w/1240}
\caption{1.png}
\end{figure}

里面的creator表示通过什么操作得到的这个Variable,grad表示反向传播的梯度

    \begin{Verbatim}[commandchars=\\\{\}]
{\color{incolor}In [{\color{incolor}29}]:} \PY{k+kn}{from} \PY{n+nn}{torch}\PY{n+nn}{.}\PY{n+nn}{autograd} \PY{k}{import} \PY{n}{Variable}
\end{Verbatim}


    \begin{Verbatim}[commandchars=\\\{\}]
{\color{incolor}In [{\color{incolor}49}]:} \PY{c+c1}{\PYZsh{} requires\PYZus{}grad 表示是否对其求梯度,默认是False}
         \PY{n}{x} \PY{o}{=} \PY{n}{Variable}\PY{p}{(}\PY{n}{torch}\PY{o}{.}\PY{n}{Tensor}\PY{p}{(}\PY{p}{[}\PY{l+m+mi}{3}\PY{p}{]}\PY{p}{)}\PY{p}{,} \PY{n}{requires\PYZus{}grad}\PY{o}{=}\PY{k+kc}{True}\PY{p}{)}
         \PY{n}{y} \PY{o}{=} \PY{n}{Variable}\PY{p}{(}\PY{n}{torch}\PY{o}{.}\PY{n}{Tensor}\PY{p}{(}\PY{p}{[}\PY{l+m+mi}{5}\PY{p}{]}\PY{p}{)}\PY{p}{,} \PY{n}{requires\PYZus{}grad}\PY{o}{=}\PY{k+kc}{True}\PY{p}{)}
         \PY{n}{z} \PY{o}{=} \PY{l+m+mi}{2} \PY{o}{*} \PY{n}{x} \PY{o}{+} \PY{n}{y} \PY{o}{+} \PY{l+m+mi}{4}
\end{Verbatim}


    \begin{Verbatim}[commandchars=\\\{\}]
{\color{incolor}In [{\color{incolor}50}]:} \PY{c+c1}{\PYZsh{} 对 x 和 y 分别求导}
         \PY{n}{z}\PY{o}{.}\PY{n}{backward}\PY{p}{(}\PY{p}{)}
\end{Verbatim}


    \begin{Verbatim}[commandchars=\\\{\}]
{\color{incolor}In [{\color{incolor}58}]:} \PY{c+c1}{\PYZsh{} x 的导数和 y 的导数}
         \PY{n+nb}{print}\PY{p}{(}\PY{l+s+s1}{\PYZsq{}}\PY{l+s+s1}{dz/dx: }\PY{l+s+si}{\PYZob{}\PYZcb{}}\PY{l+s+s1}{\PYZsq{}}\PY{o}{.}\PY{n}{format}\PY{p}{(}\PY{n}{x}\PY{o}{.}\PY{n}{grad}\PY{o}{.}\PY{n}{data}\PY{p}{)}\PY{p}{)}
         \PY{n+nb}{print}\PY{p}{(}\PY{l+s+s1}{\PYZsq{}}\PY{l+s+s1}{dz/dy: }\PY{l+s+si}{\PYZob{}\PYZcb{}}\PY{l+s+s1}{\PYZsq{}}\PY{o}{.}\PY{n}{format}\PY{p}{(}\PY{n}{y}\PY{o}{.}\PY{n}{grad}\PY{o}{.}\PY{n}{data}\PY{p}{)}\PY{p}{)}
\end{Verbatim}


    \begin{Verbatim}[commandchars=\\\{\}]
dz/dx: 
 2
[torch.FloatTensor of size 1]

dz/dy: 
 1
[torch.FloatTensor of size 1]


    \end{Verbatim}

    \subsubsection{神经网络部分}\label{ux795eux7ecfux7f51ux7edcux90e8ux5206}

所依赖的主要是 torch.nn 和 torch.nn.functional

torch.nn
里面有着所有的神经网络的层的操作,其用来构建网络,只有执行一次网络的运算才执行一次

torch.nn.functional 表示的是直接对其做一次向前运算操作

    \begin{Verbatim}[commandchars=\\\{\}]
{\color{incolor}In [{\color{incolor}59}]:} \PY{k+kn}{from} \PY{n+nn}{torch} \PY{k}{import} \PY{n}{nn}
         \PY{k+kn}{import} \PY{n+nn}{torch}\PY{n+nn}{.}\PY{n+nn}{nn}\PY{n+nn}{.}\PY{n+nn}{functional} \PY{k}{as} \PY{n+nn}{F}
\end{Verbatim}


    \begin{Verbatim}[commandchars=\\\{\}]
{\color{incolor}In [{\color{incolor} }]:} \PY{c+c1}{\PYZsh{} 基本的网络构建类模板}
        \PY{k}{class} \PY{n+nc}{net\PYZus{}name}\PY{p}{(}\PY{n}{nn}\PY{o}{.}\PY{n}{Module}\PY{p}{)}\PY{p}{:}
            \PY{k}{def} \PY{n+nf}{\PYZus{}\PYZus{}init\PYZus{}\PYZus{}}\PY{p}{(}\PY{n+nb+bp}{self}\PY{p}{)}\PY{p}{:}
                \PY{n+nb}{super}\PY{p}{(}\PY{n}{net\PYZus{}name}\PY{p}{,} \PY{n+nb+bp}{self}\PY{p}{)}\PY{o}{.}\PY{n+nf+fm}{\PYZus{}\PYZus{}init\PYZus{}\PYZus{}}\PY{p}{(}\PY{p}{)}
                \PY{c+c1}{\PYZsh{} 可以添加各种网络层}
                \PY{n+nb+bp}{self}\PY{o}{.}\PY{n}{conv1} \PY{o}{=} \PY{n}{nn}\PY{o}{.}\PY{n}{Conv2d}\PY{p}{(}\PY{l+m+mi}{3}\PY{p}{,} \PY{l+m+mi}{10}\PY{p}{,} \PY{l+m+mi}{3}\PY{p}{)}
                \PY{c+c1}{\PYZsh{} 具体每种层的参数可以去查看文档}
                
            \PY{k}{def} \PY{n+nf}{forward}\PY{p}{(}\PY{n+nb+bp}{self}\PY{p}{,} \PY{n}{x}\PY{p}{)}\PY{p}{:}
                \PY{c+c1}{\PYZsh{} 定义向前传播}
                \PY{n}{out} \PY{o}{=} \PY{n+nb+bp}{self}\PY{o}{.}\PY{n}{conv1}\PY{p}{(}\PY{n}{x}\PY{p}{)}
                \PY{k}{return} \PY{n}{out}
\end{Verbatim}



    % Add a bibliography block to the postdoc
    
    
    
    \end{document}
